\usepackage{url}
\usepackage{textcomp}
\usepackage{verbatim}
\usepackage{amsmath}
\usepackage{amsfonts}
\usepackage{amssymb}    % if you want extra symbols
%\usepackage{mathrsfs}
\usepackage{program}
\usepackage{newlfont}
\usepackage{rotating}
\usepackage{varioref}
\usepackage{graphicx}
\usepackage{makeidx}
\usepackage{tocbibind}
\usepackage{program}
\usepackage{import}
\usepackage{subfigure}
\usepackage{verbatim}
\usepackage{colortbl}

%\usepackage[small]{eulervm}
%\usepackage{courier} % for texttt



\usepackage{ulem} %underlines
\normalem % normal emph w/ ulem

%\usepackage[numbers,square,sort&compress]{natbib}
\usepackage[pdftex,plainpages=false,breaklinks=magenta,colorlinks=true,urlcolor=magenta,citecolor=magenta, linkcolor=magenta,bookmarks=true,bookmarksopen=true,bookmarksopenlevel=0,pdfstartview=Fit,pdfview=Fit,pagebackref,linktocpage=true,bookmarksnumbered=true]{hyperref}
% \usepackage{hypernat}
\usepackage{array}
\usepackage{supertabular}




% Symbols used by the authors
\newcommand{\conv}{\curvearrowright}
\newcommand{\ttt}[1]{\texttt{#1}}
\newcommand{\vect}[1]{\begin{pmatrix}#1\end{pmatrix}}
\newcommand{\paren}[1]{\left(#1\right)}
\newcommand{\brac}[1]{\left[#1\right]}
\newcommand{\braces}[1]{\left\{#1\right\}}
\newcommand{\avector}[2]{(#1_1,#1_2,\ldots,#1_{#2})} 
\newcommand{\aset}[2]{{#1_1,#1_2,\ldots,#1_{#2}}} 
\newcommand{\ith}[1]{\ensuremath{{#1^{\textrm{th}}}}} 
\newcommand{\nd}[1]{\ensuremath{{#1^{\textrm{nd}}}}} 
%\DeclareSymbolFont{AMSb}{U}{msb}{m}{n}
%\DeclareMathSymbol{\N}{\mathbin}{AMSb}{"4E}
%\DeclareMathSymbol{\realNums}{\mathbin}{AMSb}{"52}


\newcommand{\curls}[1]{\left\{#1\right\}}
%\newcommand{\teirRegEx}{\ensuremath{\paren{\Sigma\cup\curls{\brac{\Sigma\Sigma^{\ast}\Sigma}}}\paren{\Sigma\cup\curls{.}\cup\curls{\brac{\Sigma\Sigma^{\ast}\Sigma}}}^{\ast}\paren{\Sigma\cup\curls{\brac{\Sigma\Sigma^{\ast}\Sigma}}}\cup\Sigma}}
\newcommand{\teirRegEx}{\ensuremath{\Sigma\paren{\Sigma\cup\curls{.}}\Sigma}}
\newcommand{\teiresias}{\texttt{TEIRESIAS}}
\newcommand{\Teiresias}{\texttt{TEIRESIAS}}
\newcommand{\Fasta}{FastA}
\newcommand{\fasta}{FastA}
\newcommand{\psiblast}{psi--Blast}
\newcommand{\prosite}{PROSITE}
\newcommand{\biodictionary}{Bio--Dictionary}
\newcommand{\genbank}{GENBANK}
\newcommand{\embl}{EMBL}
\newcommand{\etal}{\emph{et.\ al.\ }}
\newcommand{\sptr}{SwissProt/TrEMBL}
\newcommand{\swissp}{SWISS--PROT}
\newcommand{\swissprot}{\swissp}
\newcommand{\swissptr}{SWISS--PROTACTINIUMTrEMBL}
\newcommand{\swissprottrembl}{\swissptr}
\newcommand{\amsdb}{AMSDb}
\newcommand{\ncbi}{NCBI}
\newcommand{\blosum}{BLOSUM}
\newcommand{\pam}{PAM}
\newcommand{\oligo}{oligonucleotide}
\newcommand{\Oligo}{Oligonucleotide}
\newcommand{\blast}{BLAST}
\newcommand{\pr}[1]{\prob\left(#1\right)}
\newcommand{\prt}[1]{\prob\left(\textrm{#1}\right)}
\newcommand{\cp}[2]{\prob\left(#1\mid #2\right)}
\newcommand{\cpt}[2]{\prob\left(\textrm{#1}\mid \textrm{#2}\right)}
\newcommand{\ex}[1]{\mathbf{E}\left[#1\right]}
%\newcommand{\var}[1]{\textrm{var}\left(#1\right)}
\newcommand{\phip}[3]{\Phi\paren{\frac{#1-\paren{#2}}{#3}}}
%\newcommand{\vect}[1]{\mathbf{#1}}
\newcommand{\ten}[1]{\mathbf{#1}}
\newcommand{\pdf}[2]{p_{#1}\left(#2\right)}
\newcommand{\pmf}[2]{p_{#1}\left(#2\right)}
\newcommand{\transf}[2]{M_{#1}\left(#2\right)}
\newcommand{\expo}[1]{\exp\left[#1\right]}
\newcommand{\pd}[2]{\frac{\partial}{\partial #2}\brac{#1}}
\setlength{\extrarowheight}{3pt}
\newcommand{\marnote}[1]{\marginpar{\raggedleft\footnotesize\bfseries\hspace{0pt} #1}}

\usepackage{fancyhdr}
%\renewcommand{\chaptermark}[1]{\markboth{\textit{\chaptername}\ \thechapter.\ #1}{}}

%this defines the basic headers and footer
% styles when we use the 'fancyhdr' styles
%\lhead[\fancyplain{}{\itshape\footnotesize\thepage}]{\fancyplain{}{\itshape\footnotesize\rightmark}}
%\rhead[\fancyplain{}{\itshape\footnotesize\leftmark}]{\fancyplain{}{\itshape\footnotesize\thepage}}
%\lhead[\fancyplain{}\bfseries\thepage]{\fancyplain{}\bfseries\rightmark}
%\rhead[\fancyplain{}\bfseries\leftmark]{\fancyplain{}\bfseries\thepage}
%\pagestyle{fancyplain}
\addtolength{\headwidth}{0.5\marginparsep}
\addtolength{\headwidth}{0.5\marginparwidth}
%\renewcommand{\chaptermark}[1]{\markboth{#1}{}}
%\renewcommand{\sectionmark}[1]{\markright{\thesection\ #1}}
\lhead[\fancyplain{}{\footnotesize\thepage}]{\fancyplain{}{\footnotesize\rightmark}}
\rhead[\fancyplain{}{\footnotesize\leftmark}]{\fancyplain{}{\footnotesize\thepage}}
\cfoot{}
\cfoot{}

% Special Float captions
% Different font in captions
\newcommand{\captionfonts}{\mdseries}
\newcommand{\floatnamefonts}{\bfseries}
\makeatletter  % Allow the use of @ in command names
\long\def\@makecaption#1#2{%
  \vskip\abovecaptionskip
  \sbox\@tempboxa{{\floatnamefonts #1:~~\captionfonts #2}}%
  \ifdim \wd\@tempboxa >\hsize
    {\floatnamefonts #1: \captionfonts #2\par}
  \else
    \hbox to\hsize{\hfil\box\@tempboxa\hfil}%
  \fi
  \vskip\belowcaptionskip}
\makeatother   % Cancel the effect of \makeatletter

\makeindex
